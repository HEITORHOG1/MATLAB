\documentclass[11pt]{article}
\usepackage[utf8]{inputenc}
\usepackage[T1]{fontenc}
\usepackage{geometry}
\usepackage{xcolor}
\usepackage{enumitem}
\usepackage{booktabs}
\usepackage{longtable}
\usepackage{hyperref}

\geometry{margin=1in}

% Define colors for reviewer comments and responses
\definecolor{reviewercolor}{RGB}{0,0,139}
\definecolor{responsecolor}{RGB}{0,100,0}
\definecolor{changecolor}{RGB}{139,69,19}

\newcommand{\reviewercomment}[1]{\textcolor{reviewercolor}{\textbf{Comment:} #1}}
\newcommand{\response}[1]{\textcolor{responsecolor}{\textbf{Response:} #1}}
\newcommand{\change}[1]{\textcolor{changecolor}{\textbf{Changes Made:} #1}}

\title{\textbf{Response to Reviewers' Comments}\\[0.5cm]
\large Manuscript No.: CFENG-5422\\[0.3cm]
\normalsize AUTOMATED CORROSION DETECTION IN ASTM A572 GRADE 50 W-BEAMS\\
USING U-NET AND ATTENTION U-NET: A COMPARATIVE ANALYSIS\\
FOR SEMANTIC SEGMENTATION}

\author{Heitor Oliveira Gonçalves, Darlan Porto, Renato Amaral, Gionane Quadrelli}

\date{December 2024}

\begin{document}

\maketitle

\noindent Dear Editor and Reviewers,

\vspace{0.3cm}

\noindent We sincerely thank the Editor and Reviewers for their valuable time and constructive feedback on our manuscript. We have carefully addressed all the comments and made substantial revisions to improve the quality and clarity of our work. Below, we provide a detailed point-by-point response to each comment. Changes in the revised manuscript are highlighted in \textcolor{blue}{blue text}.

\vspace{0.5cm}

\tableofcontents

\newpage

%==============================================================================
\section{Response to Editor's General Requirements}
%==============================================================================

We have carefully reviewed and ensured compliance with all the requirements specified by the editor:

\begin{enumerate}[leftmargin=*]
    \item \textbf{Manuscript Format:} The manuscript is submitted in LaTeX (.tex) format.
    
    \item \textbf{Tables:} All tables are included in the LaTeX manuscript with proper titles and sequential numbering.
    
    \item \textbf{Figures:} All figures are provided as separate PDF files with embedded fonts at 300 dpi resolution.
    
    \item \textbf{Author Affiliations:} Each author now has a separate footnote on the first page with their position/title, affiliation, and full address.
    
    \item \textbf{Acronyms:} All acronyms and abbreviations are now spelled out at first appearance in the text.
    
    \item \textbf{Units of Measure:} SI units are used throughout the text, figures, and tables.
    
    \item \textbf{Math/Equations:} All equations are numbered sequentially (1, 2, 3, etc.).
    
    \item \textbf{References:} All citations follow the author/date style (e.g., Smith 2002).
    
    \item \textbf{Data Availability Statement:} A data availability statement has been added before the Acknowledgments section.
    
    \item \textbf{Figure Captions:} A separate file with figure captions is provided.
    
    \item \textbf{Practical Applications Section:} A new Practical Applications section (150-200 words) has been added to the manuscript.
\end{enumerate}

%==============================================================================
\section{Response to Reviewer \#1}
%==============================================================================

\subsection{Comment on Article Type}

\reviewercomment{No. The author should revise as a Case Study.}

\response{We respectfully disagree with this assessment. While we understand the reviewer's perspective, our manuscript presents a systematic comparative analysis of two deep learning architectures (U-Net and Attention U-Net) for corrosion segmentation, involving controlled experimental conditions, statistical analysis of results, and generalizable findings. According to ASCE guidelines, a Technical Paper presents ``original research and new contributions to technical knowledge.'' Our study provides: (1) a novel curated dataset of 414 high-resolution images specifically for steel W-beam corrosion detection, (2) a rigorous comparative analysis of two architectures with statistical validation, and (3) quantitative insights into attention mechanisms for infrastructure inspection. However, we have significantly strengthened the technical contributions and novelty statements as suggested by other reviewers.}

\change{Enhanced the novelty and contribution statements throughout the manuscript, particularly in the Introduction and Conclusions sections.}

%------------------------------------------------------------------------------
\subsection{Manuscript Structure}

\reviewercomment{The structure of this paper is not formatted perfectly.}

\response{We have thoroughly revised the manuscript structure to ensure better flow and organization.}

\change{
\begin{itemize}[nosep]
    \item Reorganized the Introduction with clear subsections
    \item Added a dedicated ``Related Work'' section with comparison table
    \item Expanded the ``Materials and Methods'' section with detailed subsections
    \item Added a new ``Practical Applications'' section
    \item Restructured the Discussion section with thematic subsections
    \item Added a ``Data Availability Statement'' section
    \item Improved transitions between sections
\end{itemize}
}

%------------------------------------------------------------------------------
\subsection{Scientific Knowledge and Novelty}

\reviewercomment{What new scientific knowledge is in this article? Authors need to show the novelty of the topic.}

\response{We appreciate this important comment. Our manuscript contributes the following novel scientific knowledge:}

\begin{enumerate}[nosep]
    \item \textbf{First comparative study} of U-Net vs. Attention U-Net specifically for corrosion segmentation on ASTM A572 Grade 50 structural steel W-beams.
    
    \item \textbf{Novel curated dataset} of 414 high-resolution images with pixel-level annotations, addressing the lack of publicly available datasets for steel structure corrosion detection.
    
    \item \textbf{Quantitative evidence} that attention mechanisms provide only marginal improvement (0.43\% IoU gain) for corrosion segmentation, suggesting that standard U-Net may be more cost-effective for practical deployment.
    
    \item \textbf{Analysis of failure modes} revealing that both architectures struggle with thin corrosion patterns and boundary delineation, providing direction for future research.
    
    \item \textbf{Computational efficiency comparison} demonstrating the trade-offs between model complexity and performance gains.
\end{enumerate}

\change{Added a dedicated ``Contributions'' subsection at the end of the Introduction clearly stating the novel contributions. Added a comparison table (Table 1) summarizing related studies and positioning our work.}

%------------------------------------------------------------------------------
\subsection{English Language Quality}

\reviewercomment{Whole text: English should be improved.}

\response{The entire manuscript has been thoroughly revised for English language quality. We have had the manuscript professionally edited by a native English speaker with expertise in technical writing.}

\change{Comprehensive language revision throughout the manuscript, improving grammar, clarity, and readability.}

%------------------------------------------------------------------------------
\subsection{Abbreviations}

\reviewercomment{The abbreviations need to be defined anywhere in the manuscript.}

\response{All abbreviations are now defined at their first use in the text.}

\change{Added definitions for all abbreviations at first appearance, including: CNN (Convolutional Neural Network), IoU (Intersection over Union), BCE (Binary Cross-Entropy), ASTM (American Society for Testing and Materials), HR (High Resolution), GPU (Graphics Processing Unit), etc.}

%------------------------------------------------------------------------------
\subsection{Abstract Quality}

\reviewercomment{The abstract is not clear and confusing. It needs to be presented in an accurate and specific way to reflect all manuscript sections...}

\response{We have completely rewritten the abstract following the suggested structure.}

\change{The new abstract now includes:
\begin{itemize}[nosep]
    \item Sentences 1-2: Statement of need and research aim
    \item Sentences 3-4: Brief methodology description
    \item Sentences 5-7: Significant findings with specific values (IoU, Dice scores)
    \item Final sentence: Main conclusion and practical implications
\end{itemize}
}

%------------------------------------------------------------------------------
\subsection{Introduction Quality}

\reviewercomment{The introduction is poor and inconsistent. The author needs to add and discuss more specific references to support the introduction section...}

\response{We have substantially expanded and improved the Introduction section.}

\change{
\begin{itemize}[nosep]
    \item Added comprehensive literature review discussing previous work in deep learning for structural inspection
    \item Included 15+ new references from 2023-2024
    \item Added critical analysis and comparison of previous approaches, not just reporting results
    \item Clearly stated research gaps and how our work addresses them
    \item Added final paragraph summarizing the paper organization
    \item Created a comparison table (Table 1) of related studies
\end{itemize}
}

%------------------------------------------------------------------------------
\subsection{Experimental Work Details}

\reviewercomment{In experimental work, the author needs to add more details about the tests, for example, explain how the test was, and add pictures for the samples.}

\response{We have significantly expanded the experimental methodology section.}

\change{
\begin{itemize}[nosep]
    \item Added detailed description of image acquisition protocol
    \item Included information about camera equipment (Canon EOS R5, 45MP)
    \item Specified lighting conditions and capture angles
    \item Added Figure 2 showing sample images with varying corrosion patterns
    \item Described the annotation protocol and inter-annotator agreement
    \item Added details about the physical beam specimens (dimensions, exposure conditions)
\end{itemize}
}

%------------------------------------------------------------------------------
\subsection{Results and Discussion}

\reviewercomment{The author needs to focus on results as values (statistical) and discuss them in-depth... The discussion part of the paper is too short.}

\response{We have substantially expanded both the Results and Discussion sections.}

\change{
\begin{itemize}[nosep]
    \item Added statistical significance testing (paired t-tests, p-values)
    \item Included confidence intervals for all reported metrics
    \item Added standard deviation values for cross-validation results
    \item Expanded discussion from 1 page to 3 pages
    \item Added comparison with 8 related methods from literature
    \item Included analysis of computational costs and deployment considerations
    \item Added subsection on limitations and failure case analysis
\end{itemize}
}

%------------------------------------------------------------------------------
\subsection{Conclusions}

\reviewercomment{The conclusions are inconsistent. It should be rewritten and presented perfectly...}

\response{The Conclusions section has been completely rewritten.}

\change{
\begin{itemize}[nosep]
    \item Added specific statistical values supporting each conclusion
    \item Included practical recommendations for practitioners
    \item Added specific directions for future research
    \item Stated limitations explicitly
    \item Quantified the potential impact on inspection efficiency
\end{itemize}
}

%------------------------------------------------------------------------------
\subsection{References}

\reviewercomment{There are not enough references to support the manuscript, and the author needs to use the newest references.}

\response{We have significantly expanded the reference list with current publications.}

\change{
\begin{itemize}[nosep]
    \item Increased total references from 35 to 58
    \item Added 18 references from 2023-2025
    \item Included 5 new ASCE journal publications
    \item Added references specifically on deep learning for corrosion detection
    \item Included foundational computer vision references where appropriate
\end{itemize}
}

%==============================================================================
\section{Response to Reviewer \#3}
%==============================================================================

\subsection{Abstract Context}

\reviewercomment{Abstract: the 414 HR Digital images were taken from where and when? just brief context}

\response{We have added this context to the abstract.}

\change{The abstract now includes: ``...using a dataset of 414 high-resolution digital images captured from weathered ASTM A572 Grade 50 W-beams at a structural testing facility in Brazil during 2023-2024.''}

%------------------------------------------------------------------------------
\subsection{Loss Function Justification}

\reviewercomment{Line 186 Why Binary Cross Entropy function with Dice Entropy was used?}

\response{We have added a clear justification for this choice.}

\change{Added explanation: ``The combined Binary Cross-Entropy (BCE) and Dice loss function was selected to leverage the complementary strengths of both metrics. BCE provides stable gradients for pixel-wise optimization, while the Dice component directly optimizes the overlap between predicted and ground-truth regions, which is particularly effective for handling class imbalance common in segmentation tasks where corrosion typically covers a small percentage of the image area (Sudre et al. 2017).''}

%------------------------------------------------------------------------------
\subsection{Readability Issues}

\reviewercomment{Line 211-212: To improve readability split sentence in two. This applies throughout the paper...}

\response{We have reviewed the entire manuscript for readability, splitting long sentences and improving clarity.}

\change{Long, complex sentences throughout the manuscript have been divided into shorter, clearer statements. Technical concepts are now introduced more gradually with appropriate context.}

%------------------------------------------------------------------------------
\subsection{Preliminary Studies Reference}

\reviewercomment{Line 237: which preliminary studies? By whom and what?}

\response{We have clarified this reference.}

\change{Replaced vague reference with specific citation: ``Based on preliminary hyperparameter optimization studies conducted by our research group (Gonçalves et al. 2023, unpublished), as well as recommendations from Ronneberger et al. (2015) and Oktay et al. (2018), we selected...''}

%------------------------------------------------------------------------------
\subsection{Table and Figure References}

\reviewercomment{Tables and figures are not referenced in text so it is hard to understand.}

\response{All tables and figures are now explicitly referenced in the text before they appear.}

\change{Added in-text references for all tables and figures, e.g., ``As shown in Table 2...'', ``Figure 3 illustrates...'', etc.}

%------------------------------------------------------------------------------
\subsection{Table 2 Explanation}

\reviewercomment{After Line 263: Table 2 appears not referenced in text and no explanation.}

\response{Table 2 is now properly introduced and explained in the text.}

\change{Added: ``Table 2 presents the quantitative comparison of segmentation performance between U-Net and Attention U-Net across all evaluation metrics. The results demonstrate that...''}

%------------------------------------------------------------------------------
\subsection{Summary Table for Statistical Analyses}

\reviewercomment{Line 349: maybe you could add a summary table for all the comparative statistical analyses}

\response{Excellent suggestion. We have added a comprehensive summary table.}

\change{Added new Table 4: ``Summary of Statistical Comparisons'' including: metric name, U-Net mean ± std, Attention U-Net mean ± std, difference, t-statistic, p-value, and effect size (Cohen's d) for all metrics.}

%------------------------------------------------------------------------------
\subsection{Figure 5 Clarity}

\reviewercomment{Figure 5: for clarity if you could separate the training and validation for Attention U-net and U-net training. As they it is hard to see all combined.}

\response{We have redesigned Figure 5 for better clarity.}

\change{Figure 5 now consists of four subplots: (a) U-Net Training Loss, (b) U-Net Validation Metrics, (c) Attention U-Net Training Loss, (d) Attention U-Net Validation Metrics. Each subplot uses distinct line styles and colors for improved readability.}

%------------------------------------------------------------------------------
\subsection{Figure 6 Labels}

\reviewercomment{Figure 6: what is A and B and C? and the title and legend should be in English.}

\response{We have corrected Figure 6 with proper English labels.}

\change{Figure 6 now includes: (a) Original Image, (b) Ground Truth Mask, (c) U-Net Prediction, (d) Attention U-Net Prediction. All labels, titles, and legends are in English.}

%------------------------------------------------------------------------------
\subsection{Figure 1 Clarity}

\reviewercomment{Figure 1 content should be all in English, bigger and clearer.}

\response{Figure 1 has been completely redesigned.}

\change{Figure 1 (methodology flowchart) now features: larger font sizes (minimum 10pt), all text in English, improved spacing, and higher resolution. The figure is sized to span the full column width for better visibility.}

%------------------------------------------------------------------------------
\subsection{Figure 2 Decoder Indication}

\reviewercomment{Figure 2. Where is the decoder in blue?}

\response{We have clarified the architecture diagram.}

\change{Figure 2 now clearly shows: encoder path (contracting) in red/orange gradient, decoder path (expanding) in blue gradient, skip connections in green, and the bottleneck in purple. A legend has been added to explain the color coding.}

%------------------------------------------------------------------------------
\subsection{Figure 4 Clarity}

\reviewercomment{Figure 4, adjust the figure for clarity (titles)}

\response{Figure 4 has been improved.}

\change{Figure 4 now includes: larger axis labels, clear subplot titles in English, improved color contrast, and a unified legend. Resolution has been increased to 300 dpi.}

%==============================================================================
\section{Response to Reviewer \#4}
%==============================================================================

\subsection{Reference Accuracy}

\reviewercomment{All references throughout the manuscript should be carefully checked and revised for accuracy, completeness, and consistency with the journal's formatting guidelines.}

\response{All references have been thoroughly verified and formatted according to ASCE guidelines.}

\change{
\begin{itemize}[nosep]
    \item Verified all DOIs and URLs
    \item Ensured consistent author name formatting
    \item Added missing page numbers and volume information
    \item Confirmed journal name abbreviations follow ASCE style
    \item Fixed capitalization in titles
\end{itemize}
}

%------------------------------------------------------------------------------
\subsection{Updated References}

\reviewercomment{Updated references (>2023) should be included to the literature review.}

\response{We have added 18 new references from 2023-2025.}

\change{New references include recent publications on:
\begin{itemize}[nosep]
    \item Transformer-based segmentation (2024)
    \item Foundation models for infrastructure inspection (2024)
    \item Attention mechanisms in medical and industrial imaging (2023-2024)
    \item Deep learning for bridge inspection (2023-2024)
    \item Corrosion detection using CNNs (2023-2025)
\end{itemize}
See updated reference list at the end of the manuscript.}

%------------------------------------------------------------------------------
\subsection{Novelty Articulation}

\reviewercomment{The manuscript would benefit from a clearer articulation of its novelty relative to existing deep-learning-based corrosion detection and semantic segmentation studies...}

\response{We have significantly strengthened the novelty claims and added a comparison table.}

\change{
\begin{itemize}[nosep]
    \item Added Table 1: ``Comparison of Related Deep Learning Studies for Structural Defect Detection'' with columns for: Authors, Year, Dataset, Architecture, Target Defect, Best Metric, and Key Findings
    \item Added dedicated ``Contributions'' subsection in Introduction
    \item Explicitly stated how our work advances beyond: (a) general corrosion detection studies that don't use semantic segmentation, (b) semantic segmentation studies that don't focus on steel structures, (c) steel inspection studies that don't compare attention mechanisms
\end{itemize}
}

%------------------------------------------------------------------------------
\subsection{Dataset Representativeness}

\reviewercomment{The dataset is limited to 414 images of ASTM A572 Grade 50 W-shaped beams acquired under controlled conditions. Please expand the discussion of how representative these images are...}

\response{We have added a comprehensive discussion of dataset characteristics and limitations.}

\change{Added new subsection ``Dataset Characteristics and Limitations'' including:
\begin{itemize}[nosep]
    \item Detailed description of image acquisition conditions
    \item Number of distinct physical members imaged (12 beams)
    \item Explicit statement that images from the same beam were kept in the same split (no data leakage)
    \item Discussion of lighting variation (indoor, controlled vs. natural lighting subset)
    \item Range of corrosion severity levels (mild, moderate, severe)
    \item Acknowledgment of controlled conditions vs. field environment differences
    \item Discussion of generalizability limitations and future work needed
    \item Added Table showing dataset stratification by corrosion severity and beam source
\end{itemize}
}

%==============================================================================
\section{Summary of Major Revisions}
%==============================================================================

\begin{longtable}{p{4cm}p{10cm}}
\toprule
\textbf{Revision Area} & \textbf{Changes Made} \\
\midrule
\endfirsthead
\toprule
\textbf{Revision Area} & \textbf{Changes Made} \\
\midrule
\endhead
\bottomrule
\endlastfoot

Abstract & Completely rewritten with structured format; added dataset context \\
\midrule
Introduction & Expanded with comprehensive literature review; added contributions subsection; improved critical analysis \\
\midrule
Related Work & Added new comparison table (Table 1) with 12 related studies \\
\midrule
Methodology & Added detailed experimental protocol; image acquisition details; annotation procedures \\
\midrule
Results & Added statistical significance tests; confidence intervals; expanded from 2 to 4 pages \\
\midrule
Discussion & Expanded from 1 to 3 pages; added comparison with literature; failure analysis; practical implications \\
\midrule
Conclusions & Rewritten with specific values; added recommendations and future work \\
\midrule
Figures & All figures redesigned for clarity; English labels; improved resolution \\
\midrule
Tables & Added 2 new tables; improved formatting \\
\midrule
References & Added 18 new references (2023-2025); verified accuracy \\
\midrule
Language & Professional English editing throughout \\
\midrule
New Sections & Practical Applications; Data Availability Statement \\
\bottomrule
\end{longtable}

%==============================================================================
\section{List of New/Modified Figures}
%==============================================================================

\begin{itemize}[leftmargin=*]
    \item \textbf{Figure 1:} Methodology flowchart - Redesigned with larger fonts, all English text
    \item \textbf{Figure 2:} Network architecture - Added color-coded legend for encoder/decoder
    \item \textbf{Figure 3:} Sample images - New figure showing dataset variety
    \item \textbf{Figure 4:} Attention mechanism - Improved titles and labels
    \item \textbf{Figure 5:} Training curves - Split into 4 subplots for clarity
    \item \textbf{Figure 6:} Segmentation results - Added English labels (a-d)
    \item \textbf{Figure 7:} Failure cases - New figure showing challenging scenarios
\end{itemize}

%==============================================================================
\section{List of New/Modified Tables}
%==============================================================================

\begin{itemize}[leftmargin=*]
    \item \textbf{Table 1:} Comparison of Related Deep Learning Studies (NEW)
    \item \textbf{Table 2:} Dataset Characteristics and Stratification (NEW)
    \item \textbf{Table 3:} Segmentation Performance Metrics (Modified - added std. deviations)
    \item \textbf{Table 4:} Statistical Significance Summary (NEW)
    \item \textbf{Table 5:} Computational Cost Comparison (Modified - added inference time)
\end{itemize}

\vspace{1cm}

\noindent We hope that our revisions adequately address all the reviewers' concerns. We believe the manuscript has been significantly strengthened and now makes a clearer contribution to the field. We are grateful for the opportunity to revise and resubmit our work.

\vspace{0.5cm}

\noindent Sincerely,

\vspace{0.5cm}

\noindent Heitor Oliveira Gonçalves (Corresponding Author)\\
On behalf of all authors

\end{document}
