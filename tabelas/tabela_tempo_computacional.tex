% Tabela 4: Análise de Tempo Computacional
% Gerada automaticamente pelo sistema de análise

\begin{table}[htbp]
\centering
\caption{Análise comparativa de tempo computacional e eficiência entre U-Net e Attention U-Net}
\label{tab:tempo_computacional}
\begin{tabular}{lcccc}
\toprule
\textbf{Métrica} & \textbf{U-Net} & \textbf{Attention U-Net} & \textbf{Diferença (\%)} & \textbf{p-value} \\
\midrule
\multicolumn{5}{l}{\textbf{Tempo de Processamento}} \\
Treinamento (min) & 24.6 ± 4.9 & 30.3 ± 5.0 & +23.1 & --- \\
Inferência (ms) & 77 ± 12 & 107 ± 18 & +39.1 & --- \\
Taxa (FPS) & 13.45 ± 2.63 & 9.66 ± 1.73 & -28.2 & --- \\
\midrule
\multicolumn{5}{l}{\textbf{Uso de Memória}} \\
GPU (MB) & 2208 ± 150 & 2819 ± 141 & +27.7 & --- \\
RAM (MB) & 789 ± 92 & 1177 ± 101 & +49.1 & --- \\
\midrule
\multicolumn{5}{l}{\textbf{Eficiência e Complexidade}} \\
Eficiência (FPS/GB) & 6.08 ± 0.99 & 3.44 ± 0.68 & -43.4 & --- \\
Parâmetros (M) & 23.5 & 31.0 & +31.9 & --- \\
\bottomrule
\end{tabular}
\begin{tablenotes}
\small
\item Valores apresentados como média ± desvio padrão de 30 experimentos independentes.
\item Hardware: NVIDIA RTX 3080, Intel i7-10700K, 32GB RAM.
\item Significância estatística: p < 0.05 (teste t de Student).
\item Eficiência calculada como FPS por GB de memória GPU utilizada.
\item M = milhões de parâmetros treináveis.
\end{tablenotes}
\end{table}