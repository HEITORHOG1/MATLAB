\begin{table*}[htbp]
\caption{Qualitative Comparison: Classification vs Segmentation Approaches for Corrosion Assessment}
\label{tab:approach_comparison_classification}
\centering
\small
\renewcommand{\arraystretch}{1.3}
\begin{tabular}{lll}
\hline\hline
\textbf{Criterion} & \textbf{Classification (This Work)} & \textbf{Segmentation (Previous Work)} \\
\hline
Output Type & Severity class label & Pixel-level mask \\
Spatial Information & None (image-level) & Detailed (pixel-level) \\
Accuracy (Severity) & High (94\\%) & Very High (96\\%) \\
Inference Speed & Fast (30-45 ms) & Slow (850-920 ms) \\
Speedup Factor & 15-30$\\times$ faster & Baseline \\
Memory Usage & Low (2-25M params) & High (31-34M params) \\
Training Time & Moderate (2-4 hours) & Long (8-12 hours) \\
Interpretability & Class probability & Visual mask overlay \\
Primary Use Case & Rapid screening & Detailed analysis \\
Secondary Use Case & Large-scale monitoring & Critical structures \\
Deployment & Edge devices, mobile & Workstation, cloud \\
Real-time Capability & Yes (>30 fps) & No (~1 fps) \\
\hline\hline
\end{tabular}
\normalsize
\vspace{0.2cm}
\\\textit{Note: Both approaches are complementary. Classification excels at rapid screening,}
\\\textit{while segmentation provides detailed spatial analysis. Hybrid workflows recommended.}
\end{table*}
